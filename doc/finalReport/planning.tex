\section{Project Planning}
At the very beginning of the project we decided on 5 high level requirements. These are as follows: \\
\textbf{REQ1.} Designed, Described and Implemented Search Engine Algorithm, optimized to find data relevant to user queries in tightly packed, dense technical text on external pages. \\
\textbf{REQ2.} Backend Service able to, at user request, crawl (relevant) subpages of a user given website and process them into a format defined for the Search Engine. \\
\textbf{REQ3.} Backend Functionality able to call the implemented Search Engine from REQ1 with a given user query, against the existing database from REQ2. \\
\textbf{REQ4.} Frontend Functionality with a search box and a result display page, capable of calling the Backend Functionality from REQ3 with an entered text query. \\
\textbf{REQ5.} Frontend Functionality allowing authorized and authenticated site visitors to add a new page to search, using the Backend Service from REQ2.\\ \\

These gave us a rough structure to work with. We further divided up requirements into sub-requirements (i.e. each portion of the search engine was its own requirement) to create a structure that would be easier to follow. These requirements were all placed onto our github project so that we could see them at any time. For more detail on the requirements, see our requirements document. \\
We then divided the tasks between each group member (for more detail, see the roles section). The plan for the project was to work in an agile development style, with daily standups and 2-3 week long sprints. We kept track of people's progress through tickets on github's kanban board. They could be organised by status, assignees, size, and planned due date.\\

\subsection{Timeline}
Most of first semester was dedicated to research. Each member researched topics relevant to their tasks and summarised their findings to the group. After this, we begun working on the code. First each person worked on their own tasks, regularly reporting back to the team. Roughly halfway through second semester we began putting the pieces together and making their interactions work. 
Towards the end of the semester we began integrating the backend with the frontend and optimising the engine.\\
INSERT DIAGRAM OF TIMELINE HERE 
