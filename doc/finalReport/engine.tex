\subsection{Search Engine}

Whenever a search for a query is requested, the engine itself first initiates all three components, running them in parallel to calculate similarity scores against each preprocessed page.
Once all three are complete, the engine goes through all existing pages in the database, reads their 3 individual scores and compiles them into a final similarity score.

The compilation is done using the following formula:
\[ SCORE = (R_c * M) + (1 - R_c) * ((C_f * W_m) + (1 - C_f) * (W_s)) \]

Here, \(R_c\) is an arbitrary Metadata Relevance Constant, determined by trial and error, \(M\) is the Metadata Correlation Score, \(C_f\) is the meanings processor Confidence Value, \(W_m\) is the Word Meaning Correlation Score and \(W_s\) is the Word Spelling Correlation Score.

This formula is used to obtain a number for each page.
All the available pages are then compiled into an array of objects, including the page title, URL and similarity score, which is then also used to sort the array.
The array is returned from the Firebase backend search function, and can be further processed by the frontend portion of the project.
