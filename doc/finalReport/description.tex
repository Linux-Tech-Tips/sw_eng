\section{Project Description}

The key differentiating features between the TDS Engine and most commonly used search engines are:

\begin{itemize}
  \item being optimised to process dense text and technical documentation, which makes it easier for the user to find relevant pages;
  \item adding only explicitly trusted and relevant pages, thus excluding spam and minimising irrelevant results.
\end{itemize}

In order to process the raw page data into a form that can be used by the search functions, the following data processing occurs before it is stored in the database:
\begin{enumerate}
  \item html tags, script blocks, and the header section are removed from the page, saving the title for further processing;
  \item all text is set to lower case, stop words and special characters are removed, which results in a plaintext string where words are separated by one space;
  \item IDK DUDE META DATA PROCESSING
  \item for the meaings search, each word in the page is replaced by its respective word embedding. These are vectors which represent the semantic meaning of words. 
\end{enumerate}

The search functionality itself is split up into three subsections:
\begin{enumerate}
  \item a \textbf{meanings search} based on vectors representing a word's semantic meaning;
  \item a \textbf{metadata search} based on the page's metadata;
  \item a \textbf{literal text search} based on word's spellings (this functions as a fuzzy search).
\end{enumerate}
which are then combined together to create a final similarity result between a page and a query, which is passed onto the front end. \\

The front end has the following functionalities:
\begin{enumerate}
  \item users can search for pages and get redirected to a results page, which shows a ranked list of pages similar to their query;
  \item users can add pages to the engine's database.
\end{enumerate}

These functionalities are very modular, which enabled the project work to be split up and worked on simultaneously.  
