\subsection{Text Processing}
The html pages and queries are receieved in raw string format. In order to be usable by the search functions, these strings need to be processed. \\
Html pages are processed by the function htmlToText. This takes a html string as an input and returns an array. The first element of the array is the page title and the second is the page content (the body of the page). All tags and script blocks are removed, and all html entities parsed. This is achieved by using the libraries node-html-parser and html-entities. \\ 
The pre-processed pages and the raw query text are then inputted into the function stripText. This function removes all numbers, special characters, and stop words. It consciously uses a stop word list and doesn't remove all words of length less than or equal to two due to some commands being two characters long. It then removes all excess white space, returning a string of single space separated words. The stopword removal is achieved with the library stopword. \\
After the text is in this form, it is passed into the metadata and meanings processing. \\
MARTIN DO METADATA PLEASEEEEE\\
The meanings text processor takes in data from stripPages. It loops through each word in the string and looks it up in the database of word embeddings. If it is found, it adds it to a newly constructed array where each entry is a word vector. It also increments a variable to keep track of how many words are found, which it converts to a percentage at the end of the function. It returns an array where the first entry is the percentage of words found and the second entry is the matric of word vectors.
