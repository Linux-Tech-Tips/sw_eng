\subsection{Fragmentation}
Firestore has a maximum document size of 1 MiB. This is too small for larger pages when converted to matrix form. To still be able to include large pages, a (de-)fragmentation process had to be implemented when the pages are loaded into the database and when they are fetched from the database. \\

\textbf{Fragmentation} \\
Once the text has been transformed into a matrix by stringToMatrix, it is converted back into a string. 
It is then split into pieces by every 1,048,287th character, which are written to the database. 
To enable defragmentation, any pieces after the first one are titled ID\_FragmentNum and each contains a field nextID which links to the next fragment.
If there is no next fragment, nextID is set to -1. \\

\textbf{Defragmentation}\\
Defragmentation is done in the search engine component, after the data is fetched but before it is sent to the meanings score component.
Due to how the engine works, when getting the score for each part, the entire database collection of the component is accessible in code.
With this in mind, the defragmentation code simply checks nextID for each considered page, and, for any valid values found, gets the next document with the appropriate ID.
The content of each document is appended until the last fragment is found.
