\section{Introduction}

This Software Requirements Specification (\textit{SRS}) Document describes the project requirements of the project \projectname, 
developed by the Unix User Group for the University module CT216 Software Engineering. \\

The project, currently under the working title \projectname, consists of a specialised search engine, meant mainly to make it easier to find relevant technical information online. \\

The two factors differentiating this engine from currently available products are mainly the ability to search trusted pages rather the entire internet, 
and a lookup strategy optimised specifically for both dense text and technical documentation.
This approach benefits the user in the following ways:
\begin{enumerate}
    \item adding explicitly trusted pages excludes spam and advertising (present in abundance in current search engines) and introduces an unprecedented yet easily achievable level of moderation to the search results;
    \item optimising the search engine to search both through dense text and technical information will optimally make it so the user can find the relevant parts of a document much easier than in current search engines.
\end{enumerate}

One of the main goals of the project is to serve as a learning experience in the area of software engineering. For this reason, and for the purposes of optimisation,
the most important component of this project will be making the described search engine from scratch rather than relying on existing lookup libraries. \\

The project is split into multiple distinct sections, with a conscious effort to keep everything modular, so that each team member can be responsible for only a section,
rather than having to comprehend the internal details of the entire project.
When all the sections are joined together, the final result of this project should be a website, with a frontend user interface through which users will be able to use the aforementioned search engine. \\

\subsection{Purpose}

The purpose of this SRS document is to provide a thorough overview of the project requirements for the project \projectname{}. \\

The intended audience of this document is:
\begin{itemize}
    \item the product team, for reference and as a guideline for the implementation;
    \item the people marking this project, meaning the lecturer Enda Barret and possibly Teaching Support Staff.
\end{itemize}

\subsection{Key Stakeholders}

(1) \textbf{Product Owner:} The product ownership is shared equally within the team, meaning the product owners are: Martin Klacer, Lilian Kennedy, Ethan Aguocha-Njoroge, Elisabeth Pfeiffer \\
(2) \textbf{Scrum Master:} Martin Klacer \\
(3) \textbf{Development Team:} Martin Klacer, Lilian Kennedy, Ethan Aguocha-Njoroge, Elisabeth Pfeiffer \\
(4) \textbf{Other Stakeholders:} Enda Barrett (Software Engineering Module Lecturer) \\

\subsection{Scope, Objectives and Goals}

The software in question is aimed to be a search engine optimized for technical documentation and general dense technical text. \\

The objectives of the software, as outlined in the project description, are:
\begin{itemize}
    \item to allow the user to add websites explicitly stated as trusted,
    \item to allow the user to look up information in the trusted websites using a text query.
\end{itemize}

To ensure the scope is clearly outlined, the software \textbf{will not}:
\begin{itemize}
    \item search through the entire internet or the large majority of it.
\end{itemize}

The goal currently is to implement this functionality fully and ensure it achieves acceptable performance.
At this stage, setting any other goals is unnecessary and would be counterproductive.
Thanks to the nature of this project, there is a lot of potential for improvements and upgrades, which can be explored in the future when needed rather than now. \\

An overview of the expected functionality and benefits of the software can be found in the first section of the introduction chapter. \\

An overview of how project requirements are structured and organised can be found in the first section of the requirements chapter. \\

The objectives and goals of this program are explained thoroughly in detail within the actual project requirements, which can be found in the requirements chapter. \\

\subsection{References}

There are no additional documents referenced.

\subsection{Overview}

This introduction section of the SRS contains a brief introduction to the project \projectname{} and to the document itself. \\

The remaining sections of this document contain an organised list of all project requirements, followed by a brief overview of expected timeline of the project. \\

The requirements section itself contains information about how the project requirements are organised, as well as a series of diagrams highlighting how the requirement structure is interconnected.
This is followed by the entire list of the project requirements, organised and sorted in the way outlined by the requirements section introduction. \\

