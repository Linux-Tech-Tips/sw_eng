\section{Project Requirements}

This section contains the requirements created for this project, with requirement numbers for structure and clarity when referencing, as well as details for each requirement.\\

The project requirements are organized based on their specificity. At the root of the project, 5 top-level project requirements exist, each encompassing a logical component of the project.
Each of the top-level requirements is then split into parts which are more specific. Some of the parts are also split into subparts. 
The subparts, or the parts if no subparts are present, then serve as more specific requirements, guiding the creation and implementation of the software.\\

The five present top-level requirements are as follows:
\begin{itemize}
    \item \textbf{REQ1.} Search Engine
    \item \textbf{REQ2.} Backend Page Processing
    \item \textbf{REQ3.} Backend Query Processing
    \item \textbf{REQ4.} Frontend Search
    \item \textbf{REQ5.} Frontend Adding Pages
\end{itemize}

For clarity and organization, the first section of this chapter consists of a series of diagrams, intended to clarify how the top-level requirements interact with one another,
as well as how they are organized internally.

\subsection{Diagrams}

The interconnection of the top-level requirements is highlighted in the following diagram:

\includegraphics[width=\textwidth,keepaspectratio]{reqs.png}

The following diagrams represent the internal structure of the top-level requirements, the parts which they're split into and how those parts interact.

\textbf{\\REQ1:}

\includegraphics[width=\textwidth,keepaspectratio]{req1.png}

\textbf{\\REQ2:}

\includegraphics[width=\textwidth,keepaspectratio]{req2.png}

\textbf{\\REQ3:}

\includegraphics[width=\textwidth,keepaspectratio]{req3.png}

\textbf{\\REQ4:}

PLACEHOLDER FOR IMAGE ONCE REQ4 WRITTEN OUT

\textbf{\\REQ5:}

PLACEHOLDER FOR IMAGE ONCE REQ5 WRITTEN OUT


\subsection{REQ1: Search Engine Requirements}

\subsubsection{REQ1.1: Preprocessor}

\textit{Detailed requirements about the preprocessor}


\subsubsection{REQ1.2: Word Content Processor}


\textbf{\\REQ1.2.1:} The pre-processed data from \textit{REQ1.1} are converted into a vector and a matrix. The vector conversion is based on how the word is spelt so that when the distance between the two vectors are compared, it will represent how similar the spellings of the words are. The vector represents the query and the matrix represents the text data. \par

\textbf{\\REQ1.2.2:} The text matrix from \textit{REQ1.2.1} is sent to \textit{REQ1.5}. \par

\textbf{\\REQ1.2.3:} The query vector and the text matrix (both from \textit{REQ1.2.1}) are compared for spelling similarity. \par

\textbf{\\REQ1.2.4:} The results from \textit{REQ1.2.3} are sent to \textit{REQ1.6} for query matching. \par


\subsubsection{REQ1.3: Word Meaning Processor}

\textbf{REQ1.3.1:} Utilising a language model that has been fine tuned to analysing technical documentation, process all preprocessed pages from \textbf{REQ1.1}, thereby converting the pages to lists of word embeddings for each word in the page.\par

\textbf{\\REQ1.3.2:} Insert the results from \textbf{REQ1.3.1} into \textbf{REQ1.5}.\par

\textbf{\\REQ1.3.3:} Utilising the language model from \textbf{REQ1.3.1}, process the preprocessed user queries from \textbf{REQ1.1}, converting the queries to word embeddings.\par


\subsubsection{REQ1.4: Metadata Processor}

\textit{Detailed requirements about the metadata processor}


\subsubsection{REQ1.5: Database}

\textbf{REQ1.5.1:} The Search Engine contains a database created within a fitting DBMS, as of now proposed to be MySQL.\par

\textbf{\\REQ1.5.2:} The database will consist of tables capable of storing data from \textbf{REQ1.2}, \textbf{REQ1.3} and \textbf{REQ1.4}.\par

\textbf{\\REQ1.5.3:} The database will be capable of storing a list of URLs to webpages added to the engine, with the ability to match the stored data as per \textbf{REQ1.5.2} to a certain page URL.\par

\textbf{\\REQ1.5.4:} The database will store any additional page data necessary for the frontend, figured out in \textbf{REQ4}.\par

\textbf{\\REQ1.5.5:} The database will be optimized for both efficient storage of the given data and reasonably fast lookups.\par


    
\subsubsection{REQ1.6: Query Matching and Correlation}

\textbf{REQ1.6.1:} Receive processed user query data from \textit{REQ1.1}.\par

\textbf{\\REQ1.6.2:} Obtain data from the database defined by \textit{REQ1.5} about pages the engine will use to search.\par

\textbf{\\REQ1.6.3:} Obtain correlation score between processed query data from \textit{REQ1.6.1} and page data gathered from \textit{REQ1.6.2} -- 
for now, the proposal is to use dot product similarity, however, specific implementation details are left open, as a more efficient method may be figured out further in the implementation process.\par

\textbf{\\REQ1.6.4:} Obtain a total correlation score, using a formula akin to the following: 
\[ SCORE = (R_c * M) + (1 - R_c) * ((C_f * W_m) + (1 - C_f) * (W_s)) \]
where \(R_c\) is an arbitrary Metadata Relevance Constant, \(M\) is the Metadata Correlation Score, \(C_f\) is a Confidence Value, describing how confident we are in knowing the meanings of words in the query, 
\(W_m\) is the Word Meaning Correlation Score, and \(W_s\) is the Word Spelling/Content Correlation Score -- 
for now, the proposal is to use this formula, however, it is left open, as modifications to increase effectivity can be made later during development.\par

\textbf{\\REQ1.6.5:} Return the gathered page correlation info to \textit{REQ3.3}.\par





\subsection{REQ2: Backend Page Processing Requirements}


\textbf{\\REQ2.1:} The provided website url from REQ5 is crawled  through upto a predefined limit.\par

\textbf{\\REQ2.2:} The crawled urls in REQ2.1 are scraped for data.\par

\textbf{\\REQ2.3:} The data gathered in REQ2.2 is stored in a seperate database for REQ1.\par




\subsection{REQ3: Backend Query Processing Requirements}

\textit{This section will contain the detailed requirements for component 3: backend user query processing and search engine interface}



\subsection{REQ4: Frontend Search Requirements}

\textit{This section will contain the detailed requirements for component 4: frontend search box capable of interfacing with the engine to display search results}



\subsection{REQ5: Frontend Page Add Requirements}

\textit{This section will contain the detailed requirements for component 5: frontend ability to add pages to be indexed}


